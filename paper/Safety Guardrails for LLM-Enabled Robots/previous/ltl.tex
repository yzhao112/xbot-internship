While our architecture is agnostic to any particular language for safety specifications, in this paper, we focus on specifications that can be encoded using linear temporal logic (LTL). An LTL formula $\phi$ is a logical operator defined over a so-called atomic proposition set $\mathcal{AP}$.  In the context of robotics, this set represents the actions a robot can take in its current environment as well as operators that specify the temporal and logical ordering of these actions.  Throughout this work, we use the following operators: globally ($\mathbf{G}$), finally ($\mathbf{F}$), until ($\mathbf{U}$), next ($\mathbf{X}$), negation ($!$), implies ($\rightarrow$), conjunction ($\wedge$). 

Given a concatenated string of propositions (also known as ``words'') $w = w_1 w_2 \dots w_T$, where each $w_j\in\cal{AP}$ for $j\in\{1, \dots, T\}$, we say that an LTL formula $\phi$ is satisfied by $w$ if $w$ maintains the propositional structure defined by $\phi$. In plain terms, this means that the sequence of propositions in 
$w$ adheres to the logical and temporal constraints specified by the LTL formula $\phi$, such that all conditions dictated by $\phi$ (\textit{e.g.}, global invariants, eventual occurrences, or orderings) are satisfied throughout the string.  The process of determining whether an LTL formula is satisfied (or ``accepts'') by a word is referred to as model checking.  
One way to perform model checking is to transform $\phi$ into a Buchi automaton, which is a finite state machine that defines accepting and rejecting conditions for $\phi$. A Buchi automaton is defined by the five-tuple $\mathcal{B} = (Q, q_{\text{init}}, \Sigma, \delta, F)$, which contains a finite set of states $Q$, an initial state $q_{\text{init}}\in Q$, the power set $\Sigma = 2^{\mathcal{AP}}$ of $\mathcal{AP}$, a transition function $\delta:Q\times\Sigma\to Q$, and, finally, a set of accepting states $F\subseteq Q$, \textit{i.e.}, states in $Q$ that satisfy the LTL formula $\phi$.  
For the purposes of model checking, a candidate word $w$ induces some sequence of states in $\mathcal{Q}$ (also known as a ``trace''), $q_{\text{init}}q_1 \dots q_T$, as determined by the transition function~$\delta$, and we say $\phi$ accepts $w$ if and only if the trace ends in an accepting state, \textit{i.e.,} $q_T \in F$.

% To perform model checking, an LTL formula $\phi$ may be translated into a Buchi Automaton, $\mathcal{B} = (Q, q_{init}, \Sigma, \delta, F)$, which is a finite state machine that defines accepting and rejecting conditions.
% $Q$ are the states of $\mathcal{B}$, which  are defined over propositions in $\mathcal{AP}$. 
% $\Sigma = 2^\mathcal{AP}$ is the alphabet, which contains all possible propositional combinations.
% $\delta$ is the  function describing state transitions, $\delta: Q \times \Sigma \rightarrow Q$. $F$ is the set of accepting states, \textit{i.e.,} states in $Q$ that satisfy the deriving LTL formula $\phi$.