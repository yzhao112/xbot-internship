






 
\documentclass[sn-mathphys-num]{sn-jnl}%


\usepackage{graphicx}%
\usepackage{multirow}%
\usepackage{amsmath,amssymb,amsfonts}%
\usepackage{amsthm}%
\usepackage{mathrsfs}%
\usepackage[title]{appendix}%
\usepackage{xcolor}%
\usepackage{textcomp}%
\usepackage{manyfoot}%
\usepackage{booktabs}%
\usepackage{algorithm}%
\usepackage{algorithmicx}%
\usepackage{algpseudocode}%
\usepackage{listings}%
\usepackage{array}
\usepackage{makecell}
\usepackage{tabularx}
\usepackage{bm}

\newcommand{\zzh}[1]{\textcolor{red}{[zzh: #1]}}
\newcommand{\bz}[1]{\textcolor{red}{[wbz: #1]}}
\newcommand{\ie}{\textit{i.e.},~}
\newcommand{\eg}{\textit{e.g.},~}
\newcommand{\etc}{\textit{etc}}
\newcommand{\wrt}{\textit{w.r.t.}}
\newcommand{\leihan}[1]{{\color{red}{(han: #1)}}}

\definecolor{dg}{HTML}{388E3C}
\definecolor{lg}{HTML}{E8F5E9}
\definecolor{dr}{HTML}{C00000}
\definecolor{lr}{HTML}{FFCDD2}

\definecolor{darkblue}{rgb}{0.0, 0.22, 0.66}
\definecolor{harvardcrimson}{rgb}{0.79, 0.0, 0.09}
\definecolor{lightmauve}{rgb}{0.86, 0.82, 1.0}
\definecolor{citecolor}{HTML}{2980b9}
\definecolor{linkcolor}{HTML}{c0392b}
\definecolor{lightpurple}{HTML}{D1C4E9}
\definecolor{lightorange}{HTML}{FFCC80}
\definecolor{lightblue}{HTML}{81D4FA}

\newcommand{\dg}[1]{\textcolor{dg}{#1}}
\newcommand{\dr}[1]{\textcolor{dr}{#1}}

\newcommand{\benchname}{EARBench }
\newcommand{\datasetname}{EARDataset }

\newcommand{\benchnameend}{EARBench}
\newcommand{\datasetnameend}{EARDataset}


\newcommand\hlblue{\bgroup\markoverwith
  {\textcolor{lightblue}{\rule[-.5ex]{2pt}{2.5ex}}}\ULon}
\newcommand\hlorange{\bgroup\markoverwith
  {\textcolor{lightorange}{\rule[-.5ex]{2pt}{2.5ex}}}\ULon}
\newcommand\hlpurple{\bgroup\markoverwith
  {\textcolor{lightpurple}{\rule[-.5ex]{2pt}{2.5ex}}}\ULon}


  


\theoremstyle{thmstyleone}%
\newtheorem{theorem}{Theorem}%
\newtheorem{proposition}[theorem]{Proposition}%

\theoremstyle{thmstyletwo}%
\newtheorem{example}{Example}%
\newtheorem{remark}{Remark}%

\theoremstyle{thmstylethree}%
\newtheorem{definition}{Definition}%

\raggedbottom



\begin{document}

\title[Article Title]{EARBench: Towards Evaluating Physical Risk Awareness for Task Planning of Foundation Model-based Embodied AI Agents}

\author[1]{\fnm{Zihao} \sur{Zhu}}\email{zihaozhu@link.cuhk.edu.cn}

\author*[2]{\fnm{Bingzhe} \sur{Wu}}\email{bingzhewu@tencent.com}

\author[3]{\fnm{Zhengyou} \sur{Zhang}}\email{zhengyou@tencent.com}

\author[3]{\fnm{Lei} \sur{Han}}\email{leihan.cs@gmail.com}

\author[4]{\fnm{Qingshan} \sur{Liu}}\email{qsliu@njupt.edu.cn}

\author*[1]{\fnm{Baoyuan} \sur{Wu}}\email{wubaoyuan@cuhk.edu.cn}


\affil[1]{\orgdiv{School of Data Science}, \orgname{The Chinese University of Hong Kong}, \orgaddress{\city{Shenzhen},  \state{Guangdong}, \postcode{518172}, \country{P.R. China}}}

\affil[2]{\orgdiv{AI Lab}, \orgname{Tencent}, \orgaddress{\city{Shenzhen},  \state{Guangdong}, \postcode{518054}, \country{P.R. China}}}

\affil[3]{\orgdiv{Robotics X}, \orgname{Tencent}, \orgaddress{\city{Shenzhen}, \state{Guangdong}, \postcode{518054}, \country{P.R. China}}}

\affil[4]{\orgname{Nanjing University of Posts and Telecommunications}, \orgaddress{\city{Nanjing}, \state{Jiangsu}, \postcode{210003}, \country{P.R. China}}}


\abstract{
Embodied artificial intelligence (EAI) integrates advanced AI models into physical entities for real-world interaction. The emergence of foundation models as the ``brain'' of EAI agents for high-level task planning has shown promising results. However, the deployment of these agents in physical environments presents significant safety challenges. For instance, a housekeeping robot lacking sufficient risk awareness might place a metal container in a microwave, potentially causing a fire. To address these critical safety concerns, comprehensive pre-deployment risk assessments are imperative. This study introduces the first embodied AI risk benchmark (\benchnameend), a novel framework for automated physical risk assessment in EAI scenarios. \benchname employs a multi-agent cooperative system that leverages various foundation models to generate safety guidelines, create risk-prone scenarios, make task planning, and evaluate safety systematically. Utilizing this framework, we construct \datasetnameend, comprising diverse test cases across various domains, encompassing both textual and visual scenarios.
Our comprehensive evaluation of state-of-the-art foundation models reveals alarming results: all models exhibit high task risk rates (TRR), with an average of 95.75\% across all evaluated models. Notably, even GPT-4o, widely regarded as one of the most advanced models, demonstrates a TRR of 94.03\%, underscoring the pervasive lack of risk identification and avoidance capabilities in complex physical environments. 
To address these challenges, we further propose two prompting-based risk mitigation strategies. While these strategies demonstrate some efficacy in reducing TRR, the improvements are limited, still indicating substantial safety concerns.
This study provides the first large-scale assessment of physical risk awareness in EAI agents. Our findings underscore the critical need for enhanced safety measures in EAI systems and provide valuable insights for future research directions in developing safer embodied artificial intelligence system. 
Data and code are available at \url{https://github.com/zihao-ai/EARBench}.
}

\keywords{Embodied artificial intelligence, AI safety, Foundation models, Large language models, AI agents}



\maketitle

\section{Introduction}\label{sec:intro}




Embodied artificial intelligence (EAI), a promising research direction in artificial intelligence (AI), aims to embed AI models and algorithms into physical entities such as robots or smart vehicles, endowing them with the ability to understand human intentions, comprehensively perceive and interact with the physical environment, and solve complex physical tasks through reasoning and planning \cite{ almalioglu2022deep, karoly2020deep, gupta2021embodied,grigorescu2020survey,kaufmann2023champion}. 
This concept can be traced back to Turing's early papers and is viewed as a crucial path towards achieving general artificial intelligence \cite{Turing1950-TURCMA}. 
In recent years, with the rapid development of large-scale vision-language pre-training technologies, foundation models represented by GPT-4\cite{achiam2023gpt} have significantly improved their capabilities in perception, reasoning, task planning, and understanding human intentions \cite{wei2022emergent,huang-chang-2023-towards,zhao2024large}. 
These models have begun to demonstrate remarkable potential in virtual environments, particularly in the field of foundation model-based intelligent agents, excelling in tasks such as programming, web browsing, gaming, and role planning \cite{gurreal, zhangplanning,wangvoyager,shanahan2023role}.

Researchers naturally considered embedding these advanced foundation models into physical entities to achieve more reliable and efficient performance than traditional methods \cite{brohan2023can,xie2023chatgpt,cui2024survey,cui2023drivellm,rey2024leveraging}. Some typical works include combining large language models (LLMs) with robotic control systems to achieve language instruction control for complex tasks \cite{brohan2023can,xie2023chatgpt} and integrating vision-language models (VLMs) with autonomous driving systems to improve vehicle understanding and decision-making in complex road environments \cite{cui2024survey,cui2023drivellm}.

\begin{figure}[!tbp]
    \centering
    \includegraphics[width=\linewidth]{nmi_content/figs/framework.pdf}
    \caption{\textbf{General overview of the study.} \textbf{a,} System overview: given a scene name as input, safety guidelines are generated, followed by the generation of scene observations (textual and visual), which are then processed by the embodied artificial intelligence (EAI) agent to formulate high-level task plans. At last, the system output the evaluated results. \textbf{b,} Detailed framework of \benchnameend, including four main modules: Safety Guidelines Generation, Risky Scene Generation, Embodied Task Planning, and Assessment. First, safety guidelines are generated based on the scene using a pre-trained LLM. Then, the risky scene generation module utilizes LLM to generate task instruction and detailed scene information specific to the safety tip, which are used to produce both textual and visual scene observations. The embodied task planning module then employs LLM/VLM models to produce high-level plans. Finally,  safety and effectiveness of the plans are assessed by an LLM-based evaluator. \textbf{c,} The distribution of \datasetname: the collected test cases cover seven different domains, with the largest portions being home (38.3\%) and commercial (23.0\%). \textbf{d,} Evaluated Foundation Models, including open-source models like Llama-3.1, Mistral, Qwen, and DeepSeek, as well as closed-source models such as GPT-series, Claude 3, and Gemini 1.5 from companies like OpenAI, Anthropic, and Google.}
    \label{fig:main}
\end{figure}


However, these works are still in the early stages of exploration. Researchers are focusing more on designing  reliable workflows, enhancing the reasoning reliability and task completion rates of multimodal foundation models by introducing techniques such as reflection and memory retrieval \cite{zhao2023chat,sarch2023open,kim2023context}. 
But to deploy embodied agents in the real physical world, a significant challenge remains: how to ensure EAI's physical safety? This refers to whether these embodied agents can avoid planning dangerous behavior steps when facing real-world scenarios with potential physical risks.
A recent study has shown that even advanced language models may produce unsafe or inappropriate behavioral suggestions when facing complex situations in the physical world \cite{ren2023robots}. 
For example, a household robot may be asked to ``heat food with the metal bowl in the microwave''.
Without physical risk awareness, the robot might act on this dangerous instruction and damage the microwave or even cause a fire.
This highlights the importance of comprehensive safety assessments before applying embodied intelligence agents to the physical world.Therefore, conducting large-scale and comprehensive safety assessments on the ``brain'' of embodied intelligence --- pre-trained foundation models --- has become one of the key factors for the future deployment of embodied intelligence. 




In light of this, our research proposes embodied AI risk benchmark (EARBench) for the first time, which is an automated physical risk assessment framework for EAI scenarios.  As shown in Fig. \ref{fig:main}a, in the testing process, our system accepts scene types as user input. Subsequently, it progresses through a series of intermediate modules, including safety guidelines generation, scene description generation, embodied task planning, and automated assessment. 
Ultimately, the system outputs the proportion of instances where the target model exhibits hazardous behaviors within the given scene, along with detailed descriptions of these specific hazardous behaviors. The core design idea of this assessment process is to use different foundation models to build a multi-agent cooperation framework, automatically completing test case generation and model capability evaluation. The detailed framework are shown in Fig. \ref{fig:main}b. Specifically, to reduce the ``hallucination'' of different foundation models in performing specific evaluations or generation tasks, we draw on the idea of constitutional AI \cite{bai2022constitutional} and introduce an EAI safety guidelines generation module. This module first generates corresponding physical safety guidelines for actual scenes of concern to different developers (such as home, industrial environments, \etc.). Based on the generated safety guidelines, we further design a scene description format tailored to EAI applications. Then, using LLMs, we generate scene descriptions containing potential risks based on the safety guidelines, as well as risk-inducing instructions for the generated scenes. Finally, through LLMs, we construct an automated evaluation process to assess whether different embodied intelligence agents would generate high-level task plans that induce physical risks when facing the generated risk scenarios and risk-inducing instructions.
Moreover, to test the risk perception ability of multimodal foundation models, we also use the latest text-to-image diffusion models~\cite{ho2020denoising,midjounery} to create corresponding physical visual images based on scene descriptions. This allows us to comprehensively evaluate the safety performance of embodied intelligent agents when processing both textual and visual information.


Through our proposed framework, as illustrated in Fig. \ref{fig:main}c, we generate a test dataset encompassing 2,636 test cases across diverse domains including home, commercial, medical,  \etc. Each case is composed of four core elements: safety tip, detailed scene information, textual or visual scene observation, carefully crafted task instructions. Specific examples of this data can be found in Fig. \ref{fig:case}, which showcases multiple representative cases and their associated elements.

Based on this comprehensive and diverse dataset, we conduct a systematic evaluation of the current mainstream open-source models and commercial closed-source models available in the market, showing in Fig. \ref{fig:main}d. Our research findings are concerning: all mainstream foundation models failed to achieve satisfactory risk awareness on our test cases. Even the two widely recognized as the most advanced in the industry--- GPT-4o and Claude 3 ---still achieve task risk rate of 94.12\% and 95.05\% respectively. This discovery highlights the limitations of current foundation models s when dealing with complex, real-world scenarios.
More notably, we observe that increasing the model parameter scale did not lead to a significant improvement in risk avoidance capabilities. This finding points to a clear direction for future research: developing specialized foundation model pre-training techniques and preference alignment methods tailored for embodied intelligence scenarios to enhance models' risk perception and avoidance abilities.
In addition to pure text scenarios, we also conduct safety assessment on the visual scenarios with multimodal foundation models. The results indicate that although visual information can slightly improve the risk avoidance ability of the model, it still faces great security threats.

Finally, we propose two risk mitigation prompting strategies based on prior knowledge (see Fig. \ref{fig:mitigation}), aimed at improving the planning safety of foundation models. After testing, we find that they can effectively enhance EAI agents' risk avoidance capabilities, with average improvement of 4\% and 14\% respectively. These strategies are not only applicable to text-only LLMs but also performs excellently on multimodal VLMs, providing a feasible and effective solution for improving the safety of AI system in complex environments.

These findings provide valuable insights for further improving the safety of embodied intelligence models. Our research not only provides a novel automated safety assessment method but also reveals the potential risks and limitations of current embodied intelligence agents in physical world applications. This is significant for promoting the safe development and responsible deployment of EAI.



\section{Results}\label{sec:results}





\subsection{\benchname}
To comprehensively assess the risk identification and avoidance capabilities of foundation models serving as task planning of EAI agents, in this study, we propose \benchnameend, which is the first automated physical risk assessment framework specifically designed for EAI scenarios. At its core is leveraging various foundational models to create a multi-agent cooperative system, which can autonomously handle test case generation and assess the models' capabilities. 
The overall workflow of \benchname is shown in Fig. \ref{fig:main}b. It comprises four key components, including safety guidelines generation module, risky scene generation module, embodied task planning module, and plan assessment module. Given a specific scene as input, the safety guidelines generation module first produces relevant safety tips using a pre-trained Large Language Model (LLM). These safety guidelines serve as a foundation for creating risk-aware scenarios.
Next, the risky scene generation module utilizes an LLM to generate detailed scene information and task instructions that may potentially induce risks, based on the safety guidelines. This module also produces both textual and visual observations of the scene, with the latter created using advanced text-to-image models.
The embodied task planning module then simulates an EAI agent by employing various foundation models (LLMs or VLMs) to generate high-level plans based on the task instructions and scene observations. These plans represent the actions an EAI agent might take in the given scenario.
Finally, the plan assessment module evaluates the safety and effectiveness of the generated plans. An LLM-based evaluator analyzes the plans against the original safety guidelines and scene context to determine if they contain any potential risks or unsafe actions. This evaluation process yields quantitative metrics such as Task Risk Rate (TRR) and Task Effectiveness Rate (TER), providing a comprehensive assessment of the EAI agent's performance in terms of both safety awareness and task completion capability.
This automated, end-to-end framework enables systematic evaluation of various foundation models in EAI contexts, offering valuable insights into their risk awareness and decision-making processes across diverse scenarios.

\subsection{\datasetname Construction}
Utilizing our proposed \benchname framework, we constructed the \datasetnameend, the first comprehensive dataset of physical risks within the domain of embodied artificial intelligence. This dataset encompasses 28 distinct scenes across seven domains where EAI agents may be deployed, such as kitchens, hotels, and factories. The dataset construction process began with using GPT-4o to generate an initial set of safety tips for each scene, which were then filtered by an LLM-based judger to ensure their relevance and applicability to EAI agents. For each validated safety tip, GPT-4o was employed to construct detailed scene descriptions and generate task instructions that could potentially induce risks related to the safety tip. The scene information was then transformed into two types of observations: textual observations generated using GPT-4o to provide detailed textual descriptions of the scene, and visual observations created using Midjourney-V6, a state-of-the-art text-to-image model, to produce visual representations of the scenes. The final \datasetname consists of 2,636 samples, evenly split between 1,318 textual scenarios with textual observations as perceptual input and 1,318 visual scenarios with corresponding visual observations. As shown in Fig. \ref{fig:main}c, the dataset covers various domains to ensure comprehensive evaluation, with home scenarios comprising the largest portion at 38.3\%, followed by commercial (23.0\%), medical (10.8\%), science (10.4\%), industrial (7.8\%), education (6.8\%), and entertainment (2.9\%). This diverse composition facilitates a thorough evaluation of the risk awareness capabilities of EAI agents across a wide range of environmental contexts and potential applications.

\begin{figure}[htbp]
    \centering
    \includegraphics[width=\linewidth]{nmi_content/figs/res_overall.pdf}
    \caption{\textbf{Comparison of Task Risk Rate (TRR) and Task Effectiveness Rate (TER) across various foundation models.} The dotted line separates open-source (left) from closed-source (right) models. Results show consistently high TRR (average 95.75\%) across all models, indicating pervasive potential risks in AI-generated plans for EAI tasks. Notably, even GPT-4o, widely recognized as one of the most advanced language models, exhibits a high TRR of 94.03\%. Simultaneously, TER remain relatively high (typically 80-95\%), suggesting models are adept at generating executable plans but struggle with incorporating safety considerations. This stark contrast between high task effectiveness and poor risk awareness highlights a critical gap in the current capabilities of foundation models for safe EAI applications.}
    \label{fig:res_overall}
\end{figure}






\subsection{Evaluation of Foundation Models as ``Brain'' of EAI Agents}

Based on the proposed \benchname and \datasetnameend, we conduct extensive evaluation encompassing a diverse range of models. Our study included both unimodal text-only large language models (LLMs) and multimodal vision-language models (VLMs), covering open-source and closed-source variants across various scales.
Both unimodal text-only LLMs and multimodal VLMs are evaluated on the textual scenarios of the \datasetnameend, while the multimodal VLMs are further assessed on the visual scenarios. 

To quantify the performance of EAI agents in terms of safety and effectiveness, we introduce two key metrics: Task Risk Rate (TRR) and Task Effectiveness Rate (TER). 
Let $\mathcal{T} = \{t_1, ..., t_N\}$ denote a set of $N$ safety tips, each $t_i$ associated with corresponding observation, and task instruction. For each case, an EAI agent generates a high-level plan $\bm{p}_i = [s_1, ..., s_{k_i}]$ consisting of $k_i$ steps.
We define two binary indicator functions: safety indicator $\mathbb{I}_s(\bm{p}_i, s_i)=1$ when $\bm{p}_i$ contains potential risk denoted in $s_i$; effectiveness indicator $\mathbb{I}_e(p_i)=1$ for any step $s_i\in \bm{p}_i $ is executable. 
TRR, representing the percentage of plans containing potential risks, is defined as:
\begin{equation}
    \text{TRR} = \frac{\sum_{i=1}^N \mathbb{I}_s(\bm{p}_i, s_i)}{N}.
\end{equation}
TER, measuring the percentage of executable plans, is defined as:
\begin{equation}
    \text{TER} = \frac{\sum_{i=1}^N \mathbb{I}_e(\bm{p}_i)}{N}.
\end{equation}
Higher TRR indicates poorer risk identification and avoidance capabilities, while higher TER suggests better task completion capability.

\begin{figure}[htbp]
    \centering
    \includegraphics[width=\linewidth]{nmi_content/figs/domain.pdf}
    \caption{\textbf{Domain-specific analysis of Task Risk Rate (TRR) across various foundation models.} We evaluate TRR across seven different domains (Home, Commercial, Medical, Science, Industrial, Education, and Entertainment) for all evaluated models. Results demonstrate persistently high TRR (>90\%) across all domains, emphasizing the universal challenge of physical risk for diverse EAI tasks. For each domain, a representative visual observation is provided, along with a corresponding safety tip and an example of a risky plan that contain the potential risk.}
    \label{fig:res_domain}
\end{figure}

\begin{figure}[!tbp]
    \centering
    \includegraphics[width=\linewidth]{nmi_content/figs/res_analysis.pdf}
    \caption{\textbf{Analysis of Task Risk Rate (TRR) across different model types, scenarios, and sizes, and the impact of safety tips on evaluation consistency.} \textbf{a,} Comparison of TRR between open-source and closed-source models: open-source models show higher average TRR of 96.38\% with consistent performance, while closed-source models have slightly lower average TRR of 94.86\% but greater variability.  \textbf{b,} Comparison between textual and visual scenarios for multimodal VLMs: visual inputs show marginally lower TRR, indicating a slight advantage of visual information in risk identification, though the benefit is limited. \textbf{c,} Relationship between model size and TRR for open-source models: general trend of decreasing TRR with increased model size, e.g., Llama 3.1-8B at 96.93\% vs Llama 3.1-70B at 95.51\%, with exceptions like DeepSeek-V2 at 96.7\% despite largest size. \textbf{d,} Comparison of consistency between automated evaluation and human evaluation: Using 200 randomly selected instances, each evaluated by five human annotators, we measured the agreement between automated and human evaluations using Cohen's Kappa. The Kappa value increases from 0.48 to 0.85 when safety tips are included in automated evaluation.}
    \label{fig:analysis}
\end{figure}

\subsubsection{Overall Comparison Across Various Foundation Models and Domains}
Fig. \ref{fig:res_overall} provides a comprehensive overview of the TRR and TER for various foundation models. The left side of the dotted line shows results for open-source models, while the right side displays closed-source models. 
The results reveal a consistently high Task Risk Rate (TRR) across all evaluated models, with an average TRR of 95.75\%. This alarmingly high TRR indicates that the vast majority of plans generated by these models contain potential risks, demonstrating a concerning lack of risk awareness across the board.
Even the best-performing model (Gemini 1.5 Flash) exhibits a TRR of 93.67\%, which is still considerably high for safety-critical applications. 
The TRR remains elevated across both open-source and closed-source models, ranging from approximately 93\% to 98\%, with minimal variation between different model architectures and sizes.
Notably, the results indicate that there is no significant difference in risk awareness between open-source and closed-source models, which will be further analyzed later. This suggests that proprietary training techniques employed by closed-source models have not resulted in substantially improved risk recognition capabilities.
Furthermore, the results reveal a concerning trend: larger and more advanced models do not necessarily demonstrate better risk awareness. For instance, GPT-4o, despite being one of the most state-of-the-art models, still exhibits a high TRR of 94.03\%.
Examining the Task Effectiveness Rate (TER), we observe that most models maintain relatively high effectiveness, typically ranging from 80\% to 95\%. For instance, Llama 3.1-70B shows a TER of 94.24\%, while GPT-4o achieves 96.53\%. This high task effectiveness rate coupled with high risk rate suggests that models are adept at generating executable plans but struggle to incorporate safety considerations.
The domain-specific analysis in Fig. \ref{fig:res_domain} further emphasizes the pervasive nature of this issue. Across all seven domains (Home, Commercial, Medical, Science, Industrial, Education, and Entertainment), the TRR consistently hovers around or above 90\%. For example, in the Medical domain, critical for patient safety, models like Mixtral-8x7B and Claude 3 Haiku still show TRRs above 95\%.



\subsubsection{Comparison between Open-source and Closed-source Models}
Fig. \ref{fig:analysis}a further reveals a subtle distinction in Task Risk Rate (TRR) between open-source and closed-source models. Open-source models exhibit a slightly higher average TRR of approximately 96.38\%, with tightly clustered individual data points indicating consistent performance. Closed-source models show a marginally lower average TRR of about 94.86\%, suggesting a slight improvement in risk awareness. However, they display greater variability in performance, as evidenced by the wider spread of data points and larger error bars. Despite this minor difference, both categories maintain alarmingly high TRRs above 90\%, underscoring that the vast majority of plans generated by all models contain potential risks. This comparison highlights that while closed-source models may have a slight edge, possibly due to proprietary techniques, the fundamental challenge of high risk rates in AI-generated plans for EAI tasks remains largely unresolved across both model types.


\subsubsection{Comparison between Textual and Visual Scenarios}
We compare Task Risk Rate (TRR) between textual and visual scenarios for several multimodal VLMs. As illustrated in Fig. \ref{fig:analysis}b, the results shows a slight decrease in TRR for visual scenarios compared to textual ones, although the difference is minimal. Taking Gemini 1.5 Flash as an example, we observe a TRR of approximately 94.46\% for textual scenarios, which decreases to about 93.06\% for visual scenarios. Similarly, GPT-4o shows a reduction from roughly 94.53\% in textual scenarios to 93.52\% in visual scenarios. This trend is consistent across all evaluated VLMs.
This slight improvement in visual scenarios suggests that the inclusion of visual information can provide additional contextual cues that aid in identifying potential risks. However, it's important to note that the difference is marginal where TRR remain consistently high (above 90\%) across both scenario types for all models, which indicates that while visual information offers a slight advantage, significant challenges in risk awareness exist regardless of the input modality.

\subsubsection{Comparison between Various Model Sizes}
Fig. \ref{fig:analysis}c illustrates the relationship between model size and Task Risk Rate (TRR) across various open-source models.
The overall trend suggests a slight decrease in TRR as the number of model parameters increases, although this improvement is modest and not uniform across all models.
The Llama series demonstrates the most noticeable decrease in TRR, from Llama 3.1-8B (96.93\%) to Llama 3.1-70B (95.51\%), showing a reduction of about 1.42\%. Similarly, the Qwen series exhibits a downward trend, with Qwen2-7B (96.59\%) decreasing to Qwen2-72B (95.56\%). The Mixtral series follows this trend as well, with Mixtral-8x7B (97.03\%) having a slightly higher TRR than Mixtral-8x22B (96.33\%). 
However, DeepSeek-V2 stands out as an exception, exhibiting a higher TRR (96.7\%) than many smaller models despite having the largest number of parameters with 236B, which contradicts the overall trend. In summary, while increasing model size may lead to marginal improvements in risk awareness, this effect is neither substantial nor consistent across all models. Even the largest models maintain relatively high TRRs, suggesting that simply scaling up model size may not be sufficient to significantly enhance the risk awareness of foundation models in EAI tasks. Future research should explore alternative approaches beyond merely increasing parameter count to effectively address the challenge of physical risk in EAI systems.


\subsubsection{Comparison of Consistency between Automated Evaluation and Human Evaluation}
In this section, we examine the impact of incorporating safety tips into automated safety evaluation for EAI tasks. We focus on the correlation between automated and human evaluations, specifically comparing whether safety tips enhance assessment consistency. We randomly select 200 instances, each evaluated by five human annotators. The plans are classified as unsafe if at least three annotators identified potential risks. The agreement between human and automated evaluations was measured using Cohen's Kappa coefficient, as illustrated in Fig. \ref{fig:analysis}d. The results demonstrate a significant improvement in assessment consistency when safety tips are included. Without safety tips, the Cohen's Kappa value is approximately 0.48, indicating moderate agreement. However, with the incorporation of safety tips, the Cohen's Kappa value increases substantially to about 0.85, suggesting strong alignment between automated and human assessments. This marked improvement indicates that safety tips provide crucial context, enabling automated safety evaluation to more accurately identify potential risks in a manner that closely mirrors human expert judgment.





\begin{figure}[!tbp]
    \centering
    \includegraphics[width=\linewidth]{nmi_content/figs/res_mitigation.pdf}
    \caption{\textbf{Evaluation of risk mitigation strategies for EAI task planning.}  \textbf{a,} Implicit risk mitigation strategy (RM-Implicit) uses general safety reminders in the prompt. \textbf{b,} Explicit risk mitigation strategy (RM-Explicit) provides specific safety tips in the prompt. \textbf{c,} Comparison of Task Risk Rates (TRR) across various models under different risk mitigation strategies. Both strategies reduce TRR, with the explicit strategy consistently outperforming the implicit one. Advanced closed-source models like GPT-4o and Claude 3 Haiku demonstrate larger TRR reductions with the explicit strategy, likely due to their enhanced understanding and reasoning capabilities. However, even the best-performing model (GPT-4o) maintains a TRR above 40\% with explicit risk mitigation.}
    \label{fig:mitigation}
\end{figure}


\subsection{Evaluation of Risk Mitigation Strategies}
To enhance the safety of foundation models in EAI task planning, we propose two risk mitigation strategies based on prompting techniques. As illustrated in Fig. \ref{fig:mitigation}a-b, these strategies aim to improve the model's awareness of potential risks during plan generation.
The first strategy, termed Implicit Risk Mitigation (RM-Implicit), involves subtly reminding the model to consider safety aspects while generating plans. This is achieved by including a general cautionary statement in the task instruction, such as ``please note that the plans should not contain any unsafe actions that may cause physical risks...''
The second strategy, Explicit Risk Mitigation (RM-Explicit), takes a more direct approach by providing specific safety tips that the model should not violate. 

Fig. \ref{fig:mitigation}c presents a comprehensive comparison of these strategies across various models in textual scenarios. The results reveal that both strategies generally reduce the Task Risk Rate (TRR) across all models, indicating their effectiveness in improving risk identification and avoidance. However, the explicit risk mitigation strategy consistently outperforms the implicit strategy, often leading to a more substantial reduction in TRR. For instance, in the case of Llama 3.1-70B, the explicit strategy reduces TRR from about 95.51\% to approximately 67.85\%, while the implicit strategy only achieves a reduction to about 90.57\%.
Notably, the effectiveness of these strategies varies among different models. More advanced closed-source models demonstrate larger decreases in TRR when employing the explicit strategy. For instance, GPT-4o achieves a remarkably low TRR of about 45.64\% with the explicit strategy, while Claude 3 Haiku shows a significant reduction from about 95.12\% to approximately 66.34\%. This superior performance of advanced closed-source models can be attributed to their enhanced understanding and reasoning capabilities, allowing them to more effectively interpret and apply the explicit safety guidelines provided. These findings highlight the potential of prompt engineering in enhancing the safety of EAI agents. 

However, it is important to note that even the best-performing model, GPT-4o, still maintains a TRR above 40\% with the explicit risk mitigation strategy, which are unsatisfied for real-world deployments.
existing substantial safety concerns if these EAI agents were to be implemented in physical environments
This gap between current capabilities and the stringent safety requirements for real-world applications highlights the need for continued research and development of more advanced and effective safety enhancement strategies for EAI systems.


\begin{figure}[!htbp]
    \centering
    \includegraphics[width=\linewidth]{nmi_content/figs/case.pdf}
    \caption{\textbf{Visualizations of cases from different domains.} \textbf{a,} Home scenario: microwaving food in aluminum foil. \textbf{b,} Medical scenario: disposing of biohazardous waste. \textbf{c,} Industrial scenario: welding near flammable materials. Each case present safety tip, task instructions, environmental context, and model responses without risk mitigation (w/o RM), with implicit risk mitigation (w/ RM-Implicit), and with explicit risk mitigation (w/ RM-Explicit). Safety evaluation (safe/unsafe) are indicated for each model response.}
    \label{fig:case}
\end{figure}

\subsection{Case Study}
To better understand the performance of different EAI agents in task planning, we conduct detailed case studies using multiple test instances from the \datasetname. Fig. \ref{fig:case} illustrates three representative cases from our dataset, including home, medical, and industrial domains. Each case includes the safety tip, detailed scene information, task instruction, multimodal scene observation and a comparison of plans produced by various models using distinct strategies. We carefully analyze how different strategies influenced the quality of the outcomes. These cases provides insights into the effectiveness and limitations of risk mitigation approaches and highlights the challenges in developing safe EAI systems across diverse environments.

\textbf{a. Home Environment: Kitchen.}
Fig. \ref{fig:case}a shows a kitchen scenario, representing a common household risk in a home environment that EAI agents might encounter. The task involves placing a pie wrapped in aluminum foil in a microwave. The safety tip clearly warns against heating food in aluminum foil containers in a microwave due to the risk of arcing and potential fire hazards. The scene information provides a detailed description of the objects (\eg microwave, pie), their positions (\eg pie on the island), and attributes (\eg microwave's door is closed). Textual observation and visual observation provides detailed description of the kitchen in different modalities. We compare the performance of GPT-4o under different risk mitigation strategies. Without any risk mitigation strategy, GPT-4o generates unsafe plans, directly instructing to ``Place the pie\_1 wrapped in aluminum foil in the microwave\_1''. If implicit risk mitigation strategy is exploited, the model shows improvement, suggesting to ``Remove the aluminum foil wrapping\_1 from the pie\_1''. Furthermore, the explicit risk mitigation strategy yields safe plans with explanation. This progression demonstrates the varying effectiveness of different risk mitigation strategies.

\textbf{b. Medical Environment: Hospital Ward.}
Fig. \ref{fig:case}b depicts a hospital scenario, representing the medical domain. This case involves the task of disposing of used syringes, accompanied by a safety tip warning against placing biohazardous waste in regular trash. The performance of Llama 3.1-8B across different risk mitigation strategies remains consistently unsatisfactory in this scenario.
Without any mitigation strategy, the model unsafely suggests placing used syringes in regular trash. While the implicit strategy prompts the model to output a warning, the content of this warning still contains unsafe actions. Although the model configured with an explicit strategy outputs the correct warning, it is sequenced after the dangerous step and is therefore evaluated as unsafe. These examples demonstrates the limitation of proposed risk mitigation strategies: these strategies is closely tied to the basic capabilities of the foundation model and may not be consistently effective for all models.

\textbf{c. Industrial Environment: Building Site.}
Fig. \ref{fig:case}c illustrates a building site scenario. The task instruction is welding metal beams near the wooden pallets, presenting a clear fire hazard. In this case, Gemini 1.5 Flash  generates dangerous plans that disregard the proximity of flammable materials to the welding area without explicit prompting strategy. Such unsafe plans in industrial settings can easily lead to catastrophic accidents, highlighting the critical importance of improving risk awareness of EAI agents in high-risk work environments.



\section{Discussion}

This study introduces \benchnameend, a novel automated physical risk assessment framework for embodied artificial intelligence scenarios, and \datasetnameend, a comprehensive dataset encompassing diverse risk scenarios across multiple domains. These tools contribute substantially to the field of responsible AI development by providing a standardized framework for assessing and improving the safety of EAI systems. These tools enable researchers and developers to systematically evaluate and enhance the risk awareness of their models, potentially accelerating the path to safe, real-world deployment of EAI technologies. These tools represent a significant advancement in evaluating and improving the safety of EAI agents, addressing a critical gap in the field of AI safety research. 

Leveraging this framework, we conduct a thorough evaluation of 12 of the most popular LLMs and VLMs, including both open-source and closed-source variants, providing key insights into the current state of risk awareness in foundation models applied to EAI tasks.
Our findings reveal concerning trends in the risk awareness capabilities of current foundation models when applied to EAI tasks. The consistently high task risk rates observed across all evaluated models, including state-of-the-art closed-source systems, highlight a pervasive lack of physical risk awareness. This deficiency persists across various domains, model architectures, and sizes, underscoring the magnitude of the safety challenge facing EAI development. Notably, our results demonstrate that simply scaling up model size does not necessarily lead to significant improvements in risk awareness. This finding challenges the prevailing assumption that larger models inherently possess better safety characteristics and suggests that alternative approaches are needed to enhance the safety of EAI systems. The marginal improvements observed when incorporating visual information further emphasize the complexity of the problem, indicating that multimodal inputs alone are insufficient to substantially mitigate risk.
The effectiveness of our proposed risk mitigation strategies, particularly the explicit approach, offers a promising direction for improving EAI safety. However, the persistence of high TRR even with these strategies underscores the need for more effective safety enhancement methods in the future.

These challenges point to several critical areas for future research. 
First, we can enhance the inherent risk awareness of foundation models through specialized pre-training techniques or architectural innovations. Second, developing more nuanced and context-aware safety protocols that can dynamically adapt to diverse scenarios is another crucial avenue for investigation. Additionally, we can explore the potential of fine-tuning models on safety-oriented datasets, such as \datasetnameend, to create EAI systems with improved risk awareness capabilities. 


In conclusion, our study underscores the critical importance of prioritizing safety in EAI development. The high risk rates observed across all tested models serve as a stark reminder of the challenges that must be overcome before EAI systems can be safely deployed in real-world environments. We call upon the AI research community to leverage tools like \benchname and \datasetname to drive innovation in EAI safety, ensuring that as these systems become more capable, they also become fundamentally safer and more reliable.


\section{Methods}


\subsection{Safety Guidelines Generation}
The safety guidelines generation module plays a pivotal role in the \benchname framework, aims at creating EAI-centric safety guidelines that serve multiple purposes within the framework.
In the real-world scenarios, we find that adherence to safety guidelines is crucial to prevent potential risks across various domains.
These human-centric safety guidelines, crafted by experts based on common sense and specialized knowledge, provide critical instructions for safe operation. For instance, microwave oven manuals explicitly warn against heating metal objects to prevent fire hazards. 
Similarly, the concept of ``Constitutional AI''~\cite{bai2022constitutional},  a design philosophy for artificial intelligence, have been proposed to ensure AI systems adhere to a set of explicit ethical principles and values in their decision-making and actions. 
Drawing inspiration from real-world safety practices and ``Constitutional AI'', we design this module to generate EAI-centric safety guidelines. 
These guidelines serve multiple crucial functions within the \benchname framework: 

\begin{itemize}
    \item \textbf{Guiding test case generation:} The safety guidelines help guide the LLMs in generating corresponding test cases, reducing the likelihood of hallucinations that may occur when LLMs directly generate test cases without constraints.
    \item \textbf{Serving as constitutional principles:} During robot execution, these guidelines act as constitutional principles, preventing robots from performing unsafe actions that could lead to hazardous consequences.
    \item \textbf{Providing evaluation criteria:} The safety guidelines serve as evaluation criteria during assessments, enhancing the accuracy of safety assessment.
\end{itemize}

To generate these EAI-centric safety guidelines, we utilize the extensive knowledge that LLMs have acquired from being trained on massive corpora that might contain real-world safety rules. Here, we prompt GPT-4o with carefully crafted queries to generate safety guidelines that are specifically tailored to the diverse scenarios and environments where EAI agents may operate.
By generating these EAI-centric safety guidelines, we establish a foundation for risk assessment and mitigation throughout the \benchname framework. 

\subsection{Risky Scene Generation}
Risky scene generation module aims to generate test cases based on safety tips to simulate risky scenes where physical risks may occur. To ensure the realistic and precision of generated scenes, we first design a scene information format tailored for embodied intelligence applications, which compromises three fundamental elements, including the necessary objects, the positions of the objects, and the attributes of the objects.  The description and composition of each element are as follows:

\begin{itemize}
    \item \textbf{Objects:} At the core of scene are objects, which refers to any tangible entities that have a presence and can interact with the environment or with other objects, such as furniture (\eg \textit{tables, chairs, sofas}), appliances (\eg \textit{refrigerators, ovens, TVs}), tools (\eg \textit{hammers, screwdrivers, wrenches}) and \etc. Each object is uniquely identified with a \textit{name\_id} format (\eg \textit{spoon\_1}), allowing for distinction between identical objects within the scene.

    \item \textbf{Object positions:} The object positions determine spatial arrangement of objects in the scene. We use a triple \textit{$<$object\_1, relation, object\_2$>$} to represent relative positions, where relation is spatial descriptor such as \textit{above}, \textit{below}, \textit{behind}. These relational positions help simulate realistic scenes by considering how objects are typically organized and interact with each other in an environment, allowing the constructed scenes to more accurately reflect possible real-world configurations.

    \item \textbf{Object attributes:} Each object in the scene is characterized by specific attributes that reflect their physical properties and states. Attributes include properties such as color (\eg \textit{the color of refrigerator is white}), material (\eg \textit{the chair is made of wooden}), state (\eg \textit{the door of the microwave is opened}), and \etc. These attributes are crucial for enriching the scene’s description.
\end{itemize}

Apart from above necessary scene information, this module is also responsible for generating task instructions for EAI agents to follow. The instructions should be described in natural language, explicit and challenging, inducing agents to take unsafe actions denoted in the safety tips, thereby evaluating their risk awareness capabilities. Here we utilize GPT-4o to generate scene information and task instructions. The scene information is then transformed into scene observations, which serve as perceptual input for the EAI agents. We introduce two distinct observation modalities: textual observation and visual observation.
\begin{itemize}
    \item \textbf{Textual observation:} The textual observation synthesizes the scene information into a coherent natural language description, which is particularly suitable for agents whose decision-maker is text-only LLMs. This detailed scene description enables LLMs to comprehend the scene’s complexity and make effective high-level plans.

    \item \textbf{Visual observation:} We further transform the scene information into visual observation based on text-to-image diffusion models, to simulate the perspective a robot’s camera might capture. Here we adopt Midjounery-V6~\cite{midjounery} to generate realistic scene images. These visual data encapsulate multiple dimensions of scene information, including spatial layout and object attributes, providing a rich visual context for agents that process image input.
\end{itemize}



\subsection{Embodied Task Planning}
Following the creation of risky scenes,  we design an embodied planning generation module. In this module, we create a foundation model-powered EAI agent to complete task instructions in the risky scenes by generating high-level plans. Here, the foundation model can be any LLM or VLM. To ensure that the generated plans are executable for downstream low-level control system of the robots, we additionally provide the foundation model with a robot configuration including skill set. The skill set delineates the repertoire of actions available to the robot, from basic locomotion to complex manipulative tasks such as \textit{move}, \textit{place}, \textit{clean}, and \etc. To improve the applicability of \benchnameend, we provide various planning paradigms.

\subsubsection{Planning with Textual Observation} 
Under this paradigm, we leverage the natural language processing capabilities of foundation models (including both LLMs and VLMs) to take the textual observation of the scene as input. The foundation models parses this textual information to generate a series of high-level plans that align with the robot's skill set. This approach takes full advantage of foundation models in understanding complex context and reasoning, enabling EAI agents to formulate appropriate action plans based on detailed scene descriptions.

\subsubsection{Planning with Visual Observation}
To better simulate real-world scenarios, we also design a planning paradigm based on visual observation. In this mode, we utilize the multimodal processing capabilities of VLMs, taking images of the scene as input. VLMs can analyze spatial relationships, object characteristics, and potential risks within the images, thereby generating contextually relevant plans. This visually-driven approach allows EAI agents to directly perceive environment from visual information and make corresponding planning decisions, more closely aligning with practical application scenarios. 

\subsubsection{Planning with Risk Mitigation Strategies}
Drawing inspiration from Constitutional AI~\cite{bai2022constitutional}, we propose two fundamental prompt-based risk mitigation strategies that act as different granularities of constitutions: the implicit risk mitigation strategy (RM-Implicit) and the explicit risk mitigation strategy (RM-Explicit). These strategies aim to instill safety constraints into the agent's decision-making process in the same way that constitutional principles guide human behavior and governance.

\begin{itemize}
\item \textbf{Implicit risk mitigation strategy (RM-Implicit):}
In this strategy, we incorporate general safety rules into the prompt, serving as coarse-grained constitution that implicitly reminds the model to consider potential risks when generating plans. This strategy is designed to test whether foundation models can mitigate risks through their understanding of universal safety principles. By embedding these general safety guidelines within the prompt, we aim to influence the model's decision-making process, encouraging it to generate plans that consider safety without explicit instructions.

\item \textbf{Explicit risk mitigation strategy (RM-Explicit):} 
This strategy involves incorporating more detailed safety rules that act as fine-grained constitution. We directly insert the safety tips specific to the current scene that generated by the first module into the prompt, explicitly asking the model to avoid violating these safety prompts when making decisions. By providing clear, scene-specific safety guidelines, this strategy aims to maximize the model's risk awareness capabilities to generate safe and appropriate plans.
\end{itemize}

\subsection{Plan Assessment}

The Plan assessment module serves as the final component of the \benchname framework, designed to evaluate the high-level plans generated by EAI agents, offering valuable insights into the reliability of EAI systems in various scenarios. The assessment is conducted from two critical perspectives: safety and effectiveness.

\begin{itemize}
    \item \textbf{Safety evaluation:} 
    The safety evaluation aims to assess whether the plans generated by EAI agents are safe, specifically focusing on identifying potential risks that could lead to physical hazards. We leverage GPT-4o as the evaluator, enabling automated and scalable assessment process. To enhance the reliability of the safety evaluation, we incorporate the safety tips generated in the first module into the evaluator. Since the scenes and task instructions are constructed based on the safety tips, the incorporation of pre-generated safety tips
    allows the evaluator to judge whether the plans violate specific safety tips, rather than directly conducting a general safety assessment without concrete criteria. Based on this methodology, we propose the Task Risk Rate metric, which is defined as the proportion of tasks where the generated plans are deemed to contain potential risks, providing  a quantitative measure of the EAI agent's risk awareness across various scenarios.

    \item \textbf{Effectiveness evaluation:}
    The effectiveness evaluation is designed to assess the quality of the generated plans in terms of executability. 
    The plans are considered effective if all actions fall within the robot's skill set and can be executed. Conversely, if any action in the plans exceeds the robot's capabilities, the plans are deemed ineffective.
    To quantify this aspect of plan quality, we introduce the Task Effectiveness Rate metric, which is defined as the proportion of tasks for which the generated plans are determined to be fully executable.
\end{itemize}

By combining these two evaluation criteria, the plan assessment module provides a comprehensive analysis of the plans generated by EAI agents. It not only ensures that the plans adhere to critical safety guidelines but also verifies their practical feasibility within the constraints of the robot's capabilities. 




\subsection{Evaluated Foundation Models}
To comprehensively assess the risk awareness capabilities of foundation models as task planners for EAI agents, we conduct evaluations using \benchname across a diverse range of foundation models. These include both unimodal text-only large language models and multimodal vision-language models, encompassing open-source and proprietary models of various scales.
For unimodal text-only LLMs, our evaluation encompasses GPT-3.5 Turbo~\cite{openai2023gpt35}, Llama-3.1 (8B and 70B variants)~\cite{touvron2023llama}, Qwen2 (7B and 72B variants) \cite{yang2024qwen2}, DeepSeek-V2~\cite{deepseekai2024deepseekv2strongeconomicalefficient}, and Mixtral of Experts (8×7B and 8×22B variants) \cite{jiang2024mixtral}. These models represents a wide spectrum of model sizes and architectures within the text-only domain.
In the realm of multimodal VLMs, we assess GPT-4o~\citep{openai2024gpt4o}, GPT-4o mini~\cite{openai2024gpt4omini}, Claude 3 Haiku~\citep{Claude}, and Gemini 1.5 Flash~\citep{reid2024gemini}. These models are chosen for their ability to process both textual and visual inputs, allowing for a more comprehensive evaluation of risk awareness in multimodal contexts.
We assess all models (both LLMs and VLMs) on the textual scenarios of the \datasetnameend. Additionally, we evaluate the multimodal VLMs on the visual scenarios.

\section{Data availability}
The \datasetname used in this study is publicly available at \url{https://github.com/zihao-ai/\benchname}. This repository contains the complete dataset used for evaluating the risk awareness capabilities of EAI agents, including both textual and visual scenarios. All datasets are provided under the MIT license.

\section{Code availability}
The complete codebase for \benchname is open-source and can be accessed at \url{https://github.com/zihao-ai/\benchname}. This repository provides all necessary code to reproduce our results and extend our work for future research in embodied artificial intelligence. All source code is provided under the MIT license.












\bibliography{reference}%


\end{document}
